\documentclass[10pt, compress]{beamer}

\usetheme{m}

\usepackage{booktabs}
\usepackage[scale=2]{ccicons}
\usepackage{minted}
\usepackage{algorithm}
\usepackage[noend]{algpseudocode}

\algrenewcommand\algorithmicthen{:}
\algrenewcommand\algorithmicdo{:}
\algrenewcommand\algorithmicwhile{\textbf{enquanto}}
\algrenewcommand\algorithmicreturn{\textbf{devolve}}


\newcommand*{\algrule}[1][\algorithmicindent]{%
  \makebox[#1][l]{%
    \color{gray!60}
    \hspace*{.2em}% <------------- This is where the rule starts from
    \vrule height .75\baselineskip depth .25\baselineskip
  }
}
\makeatletter
\newcount\ALG@printindent@tempcnta
\def\ALG@printindent{%
    \ifnum \theALG@nested>0% is there anything to print
    \ifx\ALG@text\ALG@x@notext% is this an end group without any text?
    % do nothing
    \else
    \unskip
    % draw a rule for each indent level
    \ALG@printindent@tempcnta=1
    \loop
    \algrule[\csname ALG@ind@\the\ALG@printindent@tempcnta\endcsname]%
    \advance \ALG@printindent@tempcnta 1
    \ifnum \ALG@printindent@tempcnta<\numexpr\theALG@nested+1\relax
    \repeat
    \fi
    \fi
}
% the following line injects our new indent handling code in place of the default spacing
\patchcmd{\ALG@doentity}{\noindent\hskip\ALG@tlm}{\ALG@printindent}{}{\errmessage{failed to patch}}
\patchcmd{\ALG@doentity}{\item[]\nointerlineskip}{}{}{} % no spurious vertical space

	\let\OldStatex\Statex
	\renewcommand{\Statex}[1][3]{%
		\setlength\@tempdima{\algorithmicindent}%
				\OldStatex\hskip\dimexpr#1\@tempdima\relax}
\makeatother
% end vertical rule patch for algorithmicx
\usemintedstyle{trac}

\usepackage{luacode}

\usepackage{tikz}
\usepackage{tikz-qtree}
\usetikzlibrary{matrix,backgrounds, decorations.pathreplacing, automata, arrows}

\tikzset{onslide/.code args={<#1>#2}{%
	\only<#1>{\pgfkeysalso{#2}} % \pgfkeysalso doesn't change the path
}}
\tikzset{
	invisible/.style={opacity=0},
	visible on/.style={alt={#1{}{invisible}}},
	alt/.code args={<#1>#2#3}{%
		\alt<#1>{\pgfkeysalso{#2}}{\pgfkeysalso{#3}} % \pgfkeysalso doesn't change the path
	},
	blink/.style={onslide={<#1> mLightBrown}},
	appear/.style={visible on=<#1->, blink=#1}
}

\tikzset{
	nodes={draw, circle, font=\scriptsize},
	sibling distance=20pt,
	edge from parent path={(\tikzparentnode) -- (\tikzchildnode)}
}

\newwrite\luadebug
\immediate\openout\luadebug luadebug.lua
\AtEndDocument{\immediate\closeout\luadebug}
\newcommand\directluadebug{\immediate\write\luadebug}

\usepackage{mathtools}

\title{Leilões com orçamento}
\subtitle{Yan Soares Couto}
\date{2017}
\institute{Instituto de Matemática e Estatística}

\begin{document}

\maketitle

\begin{frame}[fragile]
	\frametitle{Problema - Entrada}

	Dados:
	\begin{itemize}
		\item Conjunto~$I$ de~$n$ participantes.
		\item Conjunto~$J$ de~$k$ itens com um item falso~$j_0$.
		\item Preços mínimos~$r_j$ para todo~$j \in J$.
		\item Valorações~$v_{i,j}$ para todo~$(i, j) \in I \times J$.
		\item Preços máximos~$m_{i,j}$ para todo~$(i, j) \in I \times J$.
	\end{itemize}

	Tal que:
	\begin{itemize}
		\item $r_{j_0} = 0$
		\item $m_{i, j_0} = \infty$ para todo~$i \in I$.
		\item $v_{i, j_0} = 0$ para todo~$i \in I$.
	\end{itemize}

\end{frame}

\begin{frame}[fragile]
	\frametitle{Problema - Solução}

	Solução:~$(\mu, p)$ com~$\mu \in I \times J$ com cada~$i \in I$ aparecendo exatamente uma vez em~$\mu$, e cada~$j \neq j_0$ no máximo uma vez, e~$p_j \geq 0$ para todo~$j \in J$, e~$p_{j_0} = 0$.
	\vfill

	Defina a utilidade de~$i \in I$ quando compra o item~$j \in J$ como
		$$ u_{i,j}(p_j) = \left\{
			\begin{array}{ll}
				v_{i,j} - p_j & \text{se $p_j < m_{i,j}$} \\
					-\infty & \text{se $p_j \geq m_{i, j}$}
			\end{array} \right.
				$$
	\vfill

	Defina~$u_i = u_{i,j}(p_j)$ para~$j$ tal que~$(i, j) \in \mu$. Se~$j = j_0$ dizemos que~$i$ \alert{não está emparelhado}.

\end{frame}

\begin{frame}[fragile]
	\frametitle{Problema - Propriedades}

	$(\mu, p)$ é \alert{viável} se~$u_i \geq 0$ para todo~$i$ e~$p_j \geq r_j$ para todo~$(i, j) \in \mu$.
	\vfill

	$(\mu, p)$ é \alert{estável} se~$u_i \geq u_{i,j}(p_j)$ para todo~$j \in J$.
	\vfill

	$(\mu, p)$ é \alert{ótima} se é estável e, para toda solução estável~$(\mu', p')$, vale que~$u_i \geq u_i'$ para todo~$i$.
\end{frame}

\begin{frame}[fragile]
	\frametitle{Modelando como grafos}

	Defina o \alert{grafo de melhores escolhas} para preços~$p$ como~$G_p \coloneqq (I \cup J, F_p)$, onde~$(i, j) \in F_p$ sse~${u_{i, j}(p_j) \geq u_{i, j'}(p_{j'})}$ para todo~$j'$.
	\vfill

	Defina o \alert{grafo de melhores escolhas viáveis} para preços~$p$ como~$\tilde{G}_p \coloneqq (I \cup J, \tilde{F}_p)$, onde~$(i, j) \in \tilde{F}_p$ sse~$(i, j) \in F_p$ e~$p_j \geq r_j$.
	\vfill

	Considere~$\hat{\mu} \coloneqq \{(i, j) \in \mu \mid j \neq j_0\}$. Note que~$\hat{\mu}$ é um emparelhamento nesses grafos. Um \alert{caminho alternante} em~$\tilde{F}_p$ é um caminho que alterna entre arestas emparelhadas e não emparelhadas.

\end{frame}

\begin{frame}[fragile]
	\frametitle{Árvore alternante}

	Uma \alert{árvore alternante} é uma árvore na qual cada caminho da raiz a uma folha é um caminho alternante. Tal árvore é \alert{maximal} se não pode ser extendida.

	Nesse caso, cada folha é o item falso, um item sem comprador, ou um participante cujas melhores escolhas viáveis já estão na árvore.

	\begin{figure}
		\begin{minipage}{0.4\textwidth}
			\begin{tikzpicture}
				\pgfmathsetmacro{\rs}{2.5}
				\foreach \i in {1,...,6} { \pgfmathsetmacro{\p}{(6-\i)*.8} \draw (0, \p) node(i\i){$i_\i$}; }
				\foreach \j in {1,...,5} { \pgfmathsetmacro{\p}{(6-\j)*.8} \draw (\rs, \p) node(j\j){$j_\j$}; }
				\draw (\rs, 0) node(j0){$j_0$};
				\foreach \a/\b in {1/1,1/2,2/3,2/4,3/0,6/0, 4/4} { \draw (i\a) edge (j\b); }
				\foreach \a/\b in {2/1,3/2,5/0} { \draw (i\a) edge[very thick] (j\b); }
				\foreach \a/\b in {3/5} { \draw (i\a) edge[dashed] (j\b); }

			\end{tikzpicture}
		\end{minipage}
		\begin{minipage}{0.4\textwidth}
			\begin{tikzpicture}
				\Tree [.{$i_1$} [.{$j_1$} \edge[very thick]; [.{$i_2$} $j_3$ $j_4$ ] ] [.{$j_2$} \edge[very thick]; [.{$i_3$} [.{$j_0$} ] ] ] ]
			\end{tikzpicture}
		\end{minipage}
		\begin{minipage}{0.18\textwidth}
			\begin{tikzpicture}[nodes={draw=none}]
				\draw (0, 0)  edge[very thick] (.4, 0);
				\draw (.4, 0) node[right]{em~$\tilde{F}_p \cap \mu$};
				\draw (0, -.5)  edge (.4, -.5);
				\draw (.4, -.5) node[right]{em~$\tilde{F}_p \setminus \mu$};
				\draw (0, -1)  edge[dashed] (.4, -1);
				\draw (.4, -1) node[right]{em~$F_p \setminus \tilde{F}_p$};
				\draw (-.25, .25) rectangle (2, -1.25);
			\end{tikzpicture}
		\end{minipage}
	\end{figure}
\end{frame}

\begin{frame}[fragile]
	\frametitle{Algoritmo - Ideia geral}

	\begin{enumerate}
		\item Começamos com~$(\mu, p) = (\emptyset, 0)$.
		\item \label{it:loop} Escolha~$i$ exposto, e encontre uma árvore alternante maximal em~$\tilde{G}_p$.
		\item Se~$j_0$ ou um item não comprado está na árvore, aumente o emparelhamento.
		\item Caso contrário, aumente os pesos de~$F_p(T)$, onde~$T$ são os participantes na árvore, e volte para o passo~\ref{it:loop}.
	\end{enumerate}
\end{frame}

\renewcommand{\d}{\delta}
\newcommand{\din}{\delta_\mathit{in}}
\newcommand{\dout}{\delta_\mathit{out}}
\newcommand{\dfsb}{\delta_\mathit{fsb}}

\begin{frame}[fragile]
	\frametitle{Algoritmo - Aumentando preços}


	Aumentaremos todos os pesos em~$F_p(T)$ por~$\d = \min(\din, \dfsb, \dout)$, onde
	\vfill

	$\din$ é o mínimo que devemos aumentar para criar uma nova primeira escolha.
	\onslide<2->{$\din = \min\{u_i - v_{i, j} + p_j \mid i \in T,\ j \notin F_p(i)\}$}
	\vfill

	$\dfsb$ é o mínimo que devemos aumentar para tornar uma aresta de primeira escolha em viável.
	\onslide<2->{$\dfsb = \min\{r_j - p_j \mid j \notin F_p(T)\}$}
	\vfill

	$\dout$ é o mínimo que devemos aumentar para que uma aresta de primeira escolha de~$T$ saia do grafo.
	\onslide<2->{$\dout = \min\{m_{i, j} - p_j \mid i \in T,\ j \in F_p(i)\}$}

\end{frame}

\begin{frame}[fragile]
	\frametitle{Algoritmo - Emparelhamentos}

	Não existe item em~$F_p(T)$ emparelhado com participante fora de~$T$.
	\vfill

	\begin{figure}
		\begin{minipage}{0.45\textwidth}
			Caso 1: \\[2ex]
			\begin{tikzpicture}
				\draw (0, 3) node(A){$i$};
				\draw (3, 2) node(B){$j$};
				\draw (0, 1) node(C){$i'$};
				\draw (A) edge (B);
				\draw (C) edge[very thick] (B);
			\end{tikzpicture}
			$$ i \in T \implies i' \in T $$
		\end{minipage}
		\begin{minipage}{0.45\textwidth}
			Caso 2: \\[2ex]
			\begin{tikzpicture}
				\draw (0, 3) node(A){$i$};
				\draw (3, 2) node(B){$j$};
				\draw (0, 1) node(C){$i'$};
				\draw (A) edge[dashed] (B);
				\draw (C) edge[very thick] (B);
			\end{tikzpicture}
			$$ i \in T \implies \bot $$
		\end{minipage}
	\end{figure}
\end{frame}

\begin{frame}[fragile]
	\frametitle{Algoritmo - Pseudocódigo}
	\begin{algorithmic}[1]
		\State Inicialize~$(\mu, p) = (\emptyset, 0)$
		\While{existe~$i_0$ não emparelhado}
			\State Seja~$T$ e~$S$ o conjunto de participantes e itens de uma
			\Statex[2] árvore alternante máxima em~$\tilde{G}_p$.
			\While{todos~$j \in S$ estão emparelhados e~$j_0 \notin S$}
				\State Calcule~$\d = \min(\din, \dfsb, \dout)$.
				\State Faça~$p_j = p_j + \d$ para todo~$j \in F_p(T)$.
				\State $\mu = \mu \cap \tilde{F}_p$
				\State Seja~$T$ e~$S$ o conjunto de participantes e itens de uma
				\Statex[3] árvore alternante máxima em~$\tilde{G}_p$.
			\EndWhile
			\State Aumente~$\mu$ usando o caminho alternante na árvore.
		\EndWhile
		\State \Return $(\mu, p)$
	\end{algorithmic}
\end{frame}

\plain{}{\alert{Perguntas?}}

\end{document}
